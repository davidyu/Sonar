\title{An Intuitive Approach to 2-Dimensional Line-Circle Intersection Tests}
\author{
    David ``Desktop" Yu
}

\documentclass[12pt]{article}

\begin{document}
\maketitle

\paragraph{Introduction}
For \emph{Sonar}, I wanted to clip the world against a growing circle.
\emph{Sonar} is a game about two player-controlled submarines that are trying to find and destroy each other in the deep sea.
For each player, visibility is limited, but pings and sonars can be sent out to slowly uncover the world.
The sonar is a time-restricted expanding circular area that reveals all that lie within it.
At the time of writing, the entire world is constructed by a set of points which connect to form a large polygon.
To implement the sonar, I needed to be able to find all line segments that lie within a circle.
The problem isn't very difficult, but there were few intuitive solutions online, so I decided to share my approach.

This document is as much for any one who is curious as to my approach as it is for myself.
I very quickly forget what I have written and all the ugly cases that I have to cover, so it is good to have a document I can go back to.
Also, I find that when I have to rigorously explain my approach, I find a lot of holes in my reasoning and potential optimizations to my code.

\end{document}
This is never printed

\paragraph{Outline}
The remainder of this article is organized as follows.
Section~\ref{previous work} gives account of previous work.
Our new and exciting results are described in Section~\ref{results}.
Finally, Section~\ref{conclusions} gives the conclusions.

\section{Previous work}\label{previous work}
A much longer \LaTeXe{} example was written by Gil~\cite{Gil:02}.

\section{Results}\label{results}
In this section we describe the results.

\section{Conclusions}\label{conclusions}
We worked hard, and achieved very little.

\bibliographystyle{abbrv}
\bibliography{main}

